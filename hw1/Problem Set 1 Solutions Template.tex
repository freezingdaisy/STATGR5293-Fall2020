\documentclass{article}
\usepackage[utf8]{inputenc}
\usepackage[shortlabels]{enumitem}
\usepackage{natbib}
\usepackage{graphicx}
\usepackage{float}


\title{Problem Set 1}
\author{[Your Name Here]}
\date{[Enter the Date Here]}

\begin{document}

\maketitle

\section*{Problem 1: Exploratory Data Analysis}

\begin{enumerate}
   \item There are X rows and Y columns. \par
   The ----- function in Python was used to get the answer. 
   
   \item Use the tables below.
   
   \item Use the tables below. 
   
   \item The distributions of Original Loan Balance, Pre-Loan DTI, and Original FICO are shown in Figures 1, 2, and 3 below. Note that these are not the correct plots themselves, but the actual plots are similar. \par
   For other answers, use the tables below. \par

\end{enumerate}

   \newpage
   
   \begin{figure}[h!]
    \centering
    \includegraphics[scale=0.50]{}
    \caption{Histogram of Annual Income}
    \label{fig:annual income_hist}
    \end{figure}

   \begin{figure}[h!]
    \centering
    \includegraphics[scale=0.50]{}
    \caption{Histogram of Pre-Loan DTI}
    \label{fig:pre loan dti_hist}
    \end{figure}

   \begin{figure}[h!]
    \centering
    \includegraphics[scale=0.50]{}
    \caption{Histogram of Original FICO}
    \label{fig:original fico_hist}
    \end{figure}
   
   \newpage
 \end{enumerate}

\begin{table}
\centering
    \begin{tabular}{l|c|c|c}
        \hline
    Field Name          & (b)        & (b)            &  (c) \\
                        & Data Type  &  Numerical/Categorical?    & Units \\ \hline
    Original Loan Balance & & &  \\ \hline
    Interest Rate  & & &  \\ \hline
    Term         & & & \\  \hline
    Monthly Payment  & & & \\ \hline
    Annual Income & & & \\ \hline
    Pre-Loan DTI  & & & \\ \hline
    Original FICO & & & \\ \hline
    Number of Trade Lines &&& \\
    \ \ \ Opened (Last 12 Months) & & &  \\ \hline
    Employment Length & & &  \\ \hline
    Housing Status & & &  \\ \hline
    Loan Status & & &  \\ \hline
    \end{tabular}
\end{table}


\begin{table}[H]
\centering
    \begin{tabular}{l|c|c|c}
        \hline
    Field Name          & (d)    & (d)  & (d)\\
                       & Bounds? & Continuous/Discrete? & Number of Unique Values \\ \hline
    Annual Income & & & \\ \hline
    Pre-Loan DTI  & & & \\ \hline
    Original FICO & & & \\ \hline
    \end{tabular}
\end{table}


\newpage



\section*{Problem 2: Timeline Associated with Data}

\begin{enumerate}[(a)]
   \item The chronological order of the fields is as follows. Items within each sub-part are in no specific order. Note that the number of bullets provided under each sub-part have no bearing on the correct number of fields in each sub-part. \par
   \begin{enumerate}[(i)]
       \item First, the borrower would supply the following to LendingClub:
       \begin{itemize}
           \item 
           \item ...
       \end{itemize}
       \item Next, LendingClub would seek information about the credit bureau and FICO. 
       \begin{itemize}
           \item 
           \item ...
       \end{itemize}
       \item Then, LendingClub would set the terms of the loan, and if the borrower agrees, the loan is made.  
       \begin{itemize}
           \item 
           \item ...
       \end{itemize}
       \item After some time, LendingClub can check on the status of the loan. 
       \begin{itemize}
           \item 
           \item ...
       \end{itemize}
   \end{enumerate}
   \item 
   \item 
   \begin{enumerate}[(i)]
       \item 
       \item 
   \end{enumerate}
 \end{enumerate}
 

 
\newpage
\section*{Problem 3: Understanding Original FICO, Pre-Loan, DTI, and Income}

\begin{enumerate}[(a)]
   \item The seaborn plot is shown in Figure 4. 
   
   \begin{figure}[h!]
    \centering
    \includegraphics[scale=0.75]{}
    \caption{Plotting FICO, DTI, and prevalance}
    \label{fig:seabornplot}
    \end{figure}
   
   \item
   \begin{enumerate}[(i)]
       \item 
       
       \item 
   \end{enumerate}
 \end{enumerate}
\end{document}
